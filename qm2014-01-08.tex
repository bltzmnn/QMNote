\documentclass[13pt,a4paper]{article}
\usepackage{braket}
\usepackage{CJK}
\bfseries  %\mdseries %字体中等权重 %\bfseries 切换到粗体
\usepackage{geometry}
\geometry{left=3cm,right=3cm,top=3cm,bottom=3cm}
\usepackage{amsmath}
\usepackage{amssymb}
\usepackage{fancyhdr}
\pagestyle{fancy} %页眉和页脚
\begin{document}
\begin{CJK}{UTF8}{song} 
\LARGE
\cfoot{\thepage}
\rfoot{sina@bltzmnn} %定义页脚
\lhead{本书属开源图书,请勿用于商业用途!} %定义页眉
\title{QM笔记}
\author{sina@bltzmnn}
\date{}
\maketitle
\newpage

\section{前言}
  写这个笔记的目的是,也极有可能是和量子力学say goodbye。2014年,本人考的是CAS的量子力学,还有普物甲。Passing it or not,it is a problem!扯远了。。。 
  
 \ \  网上所有中文的量子力学书和帖子等资料能搜索到的,我几乎都看过。有些书,我看完就删掉了,还留着的都已传到百度网盘,大家可以去下载。下面,简单介绍下。入门书个人推荐《量子力学史话》。它从光的波动性和粒子性这一对矛盾说起(唯物辩证法!!!),然后是五个基本原理和三套理论(波动,矩阵,路径积分)的发展,之后是一些悖论和新进展,如薛定谔的猫和平行宇宙等等,非常通俗易懂的一本量子科普书。朗道的书的前几章(量子力学的一些基本概念)讲解的很好,他的书,霸气和才气并存,一个字,牛!Griffiths有本入门级的书,有中文译本,挺简单的(虽然简单,我也有不明白的地方啊...)。曾谨言的书是大纲指定的,必须认真看。英文的one hundred years of quantum mechanics对复试可能有帮助,中文在下面有介绍。喀兴林的高等量子力学和刘家谟翻译的这两本书可用来理解希尔伯特空间和五个基本原理(具体的学习时,网上公认的方法是先学其数学工具——希尔伯特空间.)。哥本哈根考释可配合狄拉克的原理一起看,狄拉克的《量子力学原理》有中译本,不用全看,看看前几章,熟悉下哥本哈根解释即可。钱伯初的有些地方讲的比较直观,如中心力场。张永德的《题典》和陈鄂生的解答最好买实体书。赵凯华的新概念物理教程(量子物理)偏原子物理,侧重于实验。还有本好书叫时间之箭,个人觉得值得一读!还有一些大家自己去看吧,地址是:http://pan.baidu.com/s/1mgz72tE
  
  视频有五种:台湾的一个老师的,钱伯初的(方法独特),苏如铿的,李俊的(外号相声大师...虽是高量,但很搞笑。学累了需要放松放松,看李俊的高量吧!!!)和弦论创始人(好像在斯坦福)之一的。前四个在优酷上就有,最后一个在网易公开课上。
  
  量子,考试得高分和学好不是一个概念。但是,我和这两类人的并集没有交集。如果没有兴趣或者强大的动力,还是不要考量子的好。CAS的量子力学,只要是大纲有的内容,啥都可能考!!!电磁场的薛定谔方程、含时微扰等等,我直接忽略的内容,今年居然出现了!准备考试的话,陈鄂生有本习题解,收录了CAS等历年真题,还有张永德的量子力学大题典。这两本书的习题弄会了,应该没啥问题了。。。学到最后大家应该对五个基本原理有深刻的体会,如牛顿力学和狭义相对论一样,由几个基本原理加数学工具便可推导整个世界!!!哈哈,吹大了!
  
  最后,说说这个笔记的内容。量子力学百年回顾,译自00年Science杂志上一篇超牛综述,one hundred years of quantum mechanics。大家应该仔细看看,收益匪浅。。。Of course,it is not my job!之后,我简单总结了下《史话》讲的三套理论的历史,前后时间不一定一致,逻辑上也比较牵强。之后就是以曾谨言的卷一\footnote{曾老的书,包罗万象!从这本书引用点,从那本书引用点,结果可想而知.}、钱伯初的和周世勋的这三本书为基础,参照上面提过的几本书,按考试大纲内容写的,仓促写完,有些只写了些思路,内容比较少,其中渗入了一些从习题总结来的经验。大家如果发现有错误或者有什么建议,请到新浪微博@bltzmnn,或者email我,bltzmnn@gmail.com。
  


\section{量子物理百年回顾}

\ \ \ D.Kleppner \& R.Jackiw

全面列举一下20世纪最有影响的科学进展应当包含广义相对论、量子力学、宇宙大爆炸、遗传密码的破译、生物进化理论和其他一些读者喜欢的课题。在这些进展当中,量子力学深层次的根本属性使得它处在一个最为独特的位置。它迫使物理学家们改造他们关于实在的观念;迫使他们重新审视事物最深层次的本性;迫使他们修正位置和速度的概念以及原因和结果的定义。

尽管量子力学是为描述远离我们的日常生活经验的抽象原子世界而创立的,但它对我们日常生活的影响无比巨大。没有量子力学作为工具,就不可能有化学、生物、医学以及其他每一个关键学科的引人入胜的进展。没有量子力学就没有全球经济可言,因为作为量子力学的产物的电子学革命将我们带入了计算机时代。同时,光子学的革命也将我们带入信息时代。量子物理的杰作改变了我们的世界,科学革命为这个世界带来了的福音,也带来了潜在的威胁。

量子力学既不象广义相对论那样来自于对引力与几何关系的光辉洞察力,也不象DNA的破译那样揭开了生物学一个新的世界的神秘面纱,它的起源不是一步到位的,是历史上少有的天才荟萃在一起共同创造了它。量子的概念如此的令人困惑以至于在引入它以后的20年中几乎没有什么根本性的进展,后来一小撮物理学家花了三年时间创立了量子力学。这些科学家为自己所做的事情所困扰,甚至有时对自己的所作所为感到失望。

或许用下面的一段观察资料能最好地描述这个至关重要但又难以捉摸的理论的独特地位:量子理论是科学史上能最精确地被实验检验的理论,是科学史上最成功的理论。量子力学深深地困扰了它的创立者,然而,直到它本质上被表述成通用形式75年后的今天,一些科学界的精英们尽管承认它强大的威力,却仍然对它的基础和基本阐释不满意。

今年是Max Planck提出量子概念100周年。在他关于热辐射的经典论文中,Planck假定振动系统的总能量不能连续改变,而是以不连续的能量子形式从一个值跳到另一个值。能量子的概念太激进了,Planck后来将它搁置下来。随后,Einstein在1905年(这一年对他来说是非凡的一年)认识到光量子化的潜在意义。不过量子的观念太离奇了,后来几乎没有根本性的进展。现代量子理论的创立则是斩新的一代物理学家花了20多年时间的结晶。

您只要看一下量子理论诞生以前的物理学就能体会到量子物理的革命性影响。1890年到1900年间的物理期刊论文基本上是关于原子光谱和物质其他一些基本的可以测量的属性的文章,如粘性、弹性、电导率、热导率、膨胀系数、折射系数以及热弹性系数等。由于维多利亚型的工作机制和精巧的实验方法的发展的刺激,知识以巨大的速度累积。

然而,在同时代人看来最显著的事情是对于物质属性的简明描述基本上是经验性的。成千上万页的光谱数据罗列了大量元素波长的精确值,但是谁都不知光谱线为何会出现,更不知道它们所传递的信息。对热导率和电导率的模型解释仅符合大约半数的事实。虽有不计其数的经验定律,但都很难令人满意。比如说,Dulong-Petit定律建立了比热和物质的原子重量的简单关系,但是它有时好使,有时不好使。在多数情况下同体积气体的质量比满足简单的整数关系。元素周期表尽管为化学的繁荣提供了关键的组织规则,但也无任何理论基础。

在众多的伟大的革命性进展中,量子力学提供了一种定量的物质理论。现在,我们原则上可以理解原子结构的每一个细节;周期表也能简单自然地加以解释;巨额的光谱排列也纳入了一个优雅的理论框架。量子力学为定量的理解分子,流体和固体,导体和半导体提供了便利。它能解释诸如超流体和超导体等怪异现象,能解释诸如中子星和Bose-Einstein凝聚(在这种现象里气体中所有原子的行为象一个单一的超大原子)等奇异的物质聚集形式。量子力学为所有的科学分支和每一项高技术提供了关键的工具。
量子物理实际上包含两个方面。一个是原子层次的物质理论:量子力学;正是它我们才能理解和操纵物质世界。另一个是量子场论,它在科学中起到一个完全不同的作用,稍后我们再回到它上面来。
 

\subsection{旧量子论}
量子革命的导火线不是对物质的研究,而是辐射问题。具体的挑战是理解黑体(即某种热的物体)辐射的光谱。烤过火的人都很熟悉这样一种现象:热的物体发光,越热发出的光越明亮。光谱的范围很广,当温度升高时,光谱的峰值从红线向黄线移动,然后又向蓝线移动(这些不是我们能直接看见的)。

结合热力学和电磁学的概念似乎可以对光谱的形状作出解释,不过所有的尝试均以失败告终。然而,Planck假定振动电子辐射的光的能量是量子化的,从而得到一个表达式,与实验符合得相当完美。但是他也充分认识到,理论本身是很荒唐的,就像他后来所说的那样:“量子化只不过是一个走投无路的做法”。

Planck将他的量子假设应用到辐射体表面振子的能量上,如果没有新秀Albert Einstein,量子物理恐怕要至此结束。 1905年,他毫不犹豫的断定:如果振子的能量是量子化的,那么产生光的电磁场的能量也应该是量子化的。尽管Maxwell理论以及一个多世纪的权威性实验都表明光具有波动性,Einstein的理论还是蕴含了光的粒子性行为。随后十多年的光电效应实验显示仅当光的能量到达一些离散的量值时才能被吸收,这些能量就像是被一个个粒子携带着一样。光的波粒二象性取决于你观察问题的着眼点,这是始终贯穿于量子物理且令人头痛的实例之一,它成为接下来20年中理论上的难题。

辐射难题促成了通往量子理论的第一步,物质悖论则促成了第二步。众所周知,原子包含正负两种电荷的粒子,异号电荷相互吸引。根据电磁理论,正负电荷彼此将螺旋式的靠近,辐射出光谱范围宽广的光,直到原子坍塌为止。
接着,又是一个新秀Niels Bohr迈出了决定性的一步。1913年,Bohr提出了一个激进的假设:原子中的电子只能处于包含基态在内的定态上,电子在两个定态之间跃迁而改变它的能量,同时辐射出一定波长的光,光的波长取决于定态之间的能量差。结合已知的定律和这一离奇的假设,Bohr扫清了原子稳定性的问题。Bohr的理论充满了矛盾,但是为氢原子光谱提供了定量的描述。他认识到他的模型的成功之处和缺陷。凭借惊人的预见力,他聚集了一批物理学家创立了新的物理学。一代年轻的物理学家花了12年时间终于实现了他的梦想。

开始时,发展Bohr量子论(习惯上称为旧量子论)的尝试遭受了一次又一次的失败。接着一系列的进展完全改变了思想的进程。
 
\subsection{量子力学史}
1923年Louis de Broglie在他的博士论文中提出光的粒子行为与粒子的波动行为应该是对应存在的。他将粒子的波长和动量联系起来:动量越大,波长越短。这是一个引人入胜的想法,但没有人知道粒子的波动性意味着什么,也不知道它与原子结构有何联系。然而de Broglie的假设是一个重要的前凑,很多事情就要发生了。

1924年夏天,出现了又一个前凑。Satyendra N. Bose提出了一种全新的方法来解释Planck辐射定律。他把光看作一种无(静)质量的粒子(现称为光子)组成的气体,这种气体不遵循经典的Boltzmann统计规律,而遵循一种建立在粒子不可区分的性质(即全同性)上的一种新的统计理论。Einstein立即将Bose的推理应用于实际的有质量的气体从而得到一种描述气体中粒子数关于能量的分布规律,即著名的Bose-Einstein分布。然而,在通常情况下新老理论将预测到原子气体相同的行为。Einstein在这方面再无兴趣,因此这些结果也被搁置了10多年。然而,它的关键思想——粒子的全同性,是极其重要的。

突然,一系列事件纷至沓来,最后导致一场科学革命。从1925年元月到1928年元月:

· Wolfgang Pauli 提出了不相容原理,为周期表奠定了理论基础。

· Werner Heisenberg、Max Born 和Pascual Jordan提出了量子力学的第一个版本,矩阵力学。人们终于放弃了通过系统的方法整理可观察的光谱线来理解原子中电子的运动这一历史目标。

· Erwin Schrödinger提出了量子力学的第二种形式,波动力学。在波动力学中,体系的状态用Schrödinger方程的解­——波函数来描述。矩阵力学和波动力学貌似矛盾,实质上是等价的。

· 电子被证明遵循一种新的统计规律,Fermi-Dirac统计。人们进一步认识到所有的粒子要么遵循Fermi-Dirac统计,要么遵循Bose-Einstein统计,这两类粒子的基本属性很不相同。

· Heisenberg阐明测不准原理。

· Paul A. M. Dirac提出了相对论性的波动方程用来描述电子,解释了电子的自旋并且预测了反物质。

· Dirac提出电磁场的量子描述,建立了量子场论的基础。

· Bohr提出互补原理(一个哲学原理),试图解释量子理论中一些明显的矛盾,特别是波粒二象性。

\ \ 量子理论的主要创立者都是年轻人。1925年,Pauli 25岁,Heisenberg和Enrico Fermi 24岁,Dirac和Jordan 23岁。\newline Schrodinger是一个大器晚成者,36岁。Born和Bohr年龄稍大一些,值得一提的是他们的贡献大多是阐释性的。Einstein的反应反衬出量子力学这一智力成果深刻而激进的属性:他拒绝自己发明的导致量子理论的许多关键的观念,他关于Bose-Einstein 统计的论文是他对理论物理的最后一项贡献,也是对物理学的最后一项重要贡献。

\ \ 创立量子力学需要新一代物理学家并不令人惊讶,Lord Kelvin在祝贺Bohr 1913年关于氢原子的论文的一封书信中表述了其中的原因。他说,Bohr的论文中有很多真理是他所不能理解的。Kelvin认为基本的新物理学必将出自无拘无束的头脑。

1928年,革命结束,量子力学的基础本质上已经建立好了。后来,Abraham Pais以轶事的方式记录了这场以狂热的节奏发生的革命。其中有一段是这样的,1925年,Samuel Goudsmit和George Uhlenbeck就提出了电子自旋的概念,Bohr对此深表怀疑。10月Bohr乘火车前往荷兰的莱顿参加Hendrik A. Lorentz的50岁生日庆典,Pauli在德国的汉堡格碰到Bohr并探询Bohr对电子自旋可能性的看法;Bohr用他那著名的低调评价的语言回答说,自旋这一提议是 “非常,非常有趣的”。后来,Einstein和Paul Ehrenfest在莱顿碰到了Bohr并讨论了自旋。Bohr说明了自己的反对意见,但是Einstein展示了自旋的一种方式并使Bohr成为自旋的支持者。在Bohr的返程中,遇到了更多的讨论者。当火车经过德国的哥挺根时,Heisenberg和Jordan接站并询问他的意见,Pauli也特意从汉堡格赶到柏林接站。Bohr告诉他们自旋的发现是一重大进步。

量子力学的创建触发了科学的淘金热。早期的成果有:1927年Heisenberg得到了氦原子Schrödinger方程的近似解,建立了原子结构理论的基础;John Slater,Douglas Rayner Hartree, 和Vladimir Fock随后又提出了原子结构的一般计算技巧;Fritz London和Walter Heitler解决了氢分子的结构,在此基础上,Linus Pauling建立了理论化学;Arnold Sommerfeld和Pauli建立了金属电子理论的基础,Felix Bloch创立了能带结构理论;Heisenberg解释了铁磁性的起因。1928年George Gamow解释了α放射性衰变的随机本性之谜,他表明α衰变是由量子力学的隧道效应引起的。随后几年中,Hans Bethe建立了核物理的基础并解释了恒星的能量来源。随着这些进展,原子物理、分子物理、固体物理和核物理进入了现代物理的时代。
 
\subsection{量子力学要点}
伴随着这些进展,围绕量子力学的阐释和正确性发生了许多争论。Bohr和Heisenberg是倡导者的重要成员,他们信奉新理论,Einstein和Schrödinger则对新理论不满意。要理解这些混乱的原因,必须掌握量子理论的关键特征,总结如下。(为了简明,我们只描述Schrödinger的波动力学。)

基本描述:波函数。系统的行为用Schrödinger方程描述,方程的解称为波函数。系统的完整信息用它的波函数表述,通过波函数可以计算任意可观察量的可能值。在空间给定体积内找到一个电子的概率正比于波函数幅值的平方,因此,粒子的位置分布在波函数所在的体积内。粒子的动量依赖于波函数的斜率,波函数越陡,动量越大。斜率是变化的,因此动量也是分布的。这样,有必要放弃位移和速度能确定到任意精度的经典图象,而采纳一种模糊的概率图象,这也是量子力学的核心。

对于同样一些系统进行同样精心的测量不一定产生同一结果,相反,结果分散在波函数描述的范围内,因此,电子特定的位置和动量没有意义。这可由测不准原理表述如下:要使粒子位置测得精确,波函数必须是尖峰型的,然而,尖峰必有很陡的斜率,因此动量就分布在很大的范围内;相反,若动量有很小的分布,波函数的斜率必很小,因而波函数分布于大范围内,这样粒子的位置就更加不确定了。

波的干涉。波相加还是相减取决于它们的相位,振幅同相时相加,反相时相减。当波沿着几条路径从波源到达接收器,比如光的双缝干涉,一般会产生干涉图样。粒子遵循波动方程,必有类似的行为,如电子衍射。至此,类推似乎是合理的,除非要考察波的本性。波通常认为是媒质中的一种扰动,然而量子力学中没有媒质,从某中意义上说根本就没有波,波函数本质上只是我们对系统信息的一种陈述。

对称性和全同性。氦原子由两个电子围绕一个核运动而构成。氦原子的波函数描述了每一个电子的位置,然而没有办法区分哪个电子究竟是哪个电子,因此,电子交换后看不出体系有何变化,也就是说在给定位置找到电子的概率不变。由于概率依赖于波函数的幅值的平方,因而粒子交换后体系的波函数与原始波函数的关系只可能是下面的一种:要么与原波函数相同,要么改变符号,即乘以-1。到底取谁呢?

量子力学令人惊诧的一个发现是电子的波函数对于电子交换变号。其结果是戏剧性的,两个电子处于相同的量子态,其波函数相反,因此总波函数为零,也就是说两个电子处于同一状态的概率为0,此即Pauli不相容原理。所有半整数自旋的粒子(包括电子)都遵循这一原理,并称为费米子。自旋为整数的粒子(包括光子)的波函数对于交换不变号,称为玻色子。电子是费米子,因而在原子中分层排列;光由玻色子组成,所以激光光线呈现超强度的光束(本质上是一个量子态)。最近,气体原子被冷却到量子状态而形成Bose-Einstein凝聚,这时体系可发射超强物质束,形成原子激光。

这一观念仅对全同粒子适用,因为不同粒子交换后波函数显然不同。因此仅当粒子体系是全同粒子时才显示出玻色子或费米子的行为。同样的粒子是绝对相同的,这是量子力学最神秘的侧面之一,量子场论的成就将对此作出解释。
 
\subsection{争议与混乱}
量子力学意味着什么?波函数到底是什么?测量是什么意思?这些问题在早期都激烈争论过。直到1930年,Bohr和他的同事或多或少地提出了量子力学的标准阐释,即哥本哈根阐释;其关键要点是通过Bohr的互补原理对物质和事件进行概率描述,调和物质波粒二象性的矛盾。Einstein不接受量子理论,他一直就量子力学的基本原理同Bohr争论,直至1955年去世。

关于量子力学争论的焦点是:究竟是波函数包含了体系的所有信息,还是有隐含的因素(隐变量)决定了特定测量的结果。60年代中期John S. Bell证明,如果存在隐变量,那么实验观察到的概率应该在一个特定的界限之下,此即Bell不等式。多数小组的实验结果与Bell不等式相悖,他们的数据断然否定了隐变量存在的可能性。这样,大多数科学家对量子力学的正确性不再怀疑了。
然而,由于量子理论神奇的魔力,它的本质仍然吸引着人们的注意力。量子体系的古怪性质起因于所谓的纠缠态,简单说来,量子体系(如原子)不仅能处于一系列的定态,也可以处于它们的叠加态。测量处于叠加态原子的某种性质(如能量),一般说来,有时得到这一个值,有时得到另一个值。至此还没有出现任何古怪。

但是可以构造处于纠缠态的双原子体系,使得两个原子共有相同的性质。当这两个原子分开后,一个原子的信息被另一个共享(或者说是纠缠)。这一行为只有量子力学的语言才能解释。这个效应太不可思议以至于只有少数活跃的理论和实验机构在集中精力研究它,论题并不限于原理的研究,而是纠缠态的用途;纠缠态已经应用于量子信息系统,也成为量子计算机的基础。
 
\subsection{二次革命}
在20年代中期创立量子力学的狂热年代里,也在进行着另一场革命,量子物理的另一个分支——量子场论的基础正在建立。不像量子力学的创立那样如暴风疾雨般一挥而就,量子场论的创立经历了一段曲折的历史,一直延续到今天。尽管量子场论是困难的,但它的预测精度是所有物理学科中最为精确的,同时,它也为一些重要的理论领域的探索提供了范例。

激发提出量子场论的问题是电子从激发态跃迁到基态时原子怎样辐射光。1916年,Einstein研究了这一过程,并称其为自发辐射,但他无法计算自发辐射系数。解决这个问题需要发展电磁场(即光)的相对论量子理论。量子力学是解释物质的理论,而量子场论正如其名,是研究场的理论,不仅是电磁场,还有后来发现的其它场。

1925年,Born,Heisenberg和Jordan发表了光的量子场论的初步想法,但关键的一步是年轻且本不知名的物理学家Dirac于1926年独自提出的场论。Dirac的理论有很多缺陷:难以克服的计算复杂性,预测出无限大量,并且显然和对应原理矛盾。

40年代晚期,量子场论出现了新的进展,Richard Feynman,Julian Schwinger和Sin-Itiro Tomonaga提出了量子电动力学(缩写为QED)。他们通过重整化的办法回避无穷大量,其本质是通过减掉一个无穷大量来得到有限的结果。由于方程复杂,无法找到精确解,所以通常用级数来得到近似解,不过级数项越来越难算。虽然级数项依次减小,但是总结果在某项后开始增大,以至于近似过程失败。尽管存在这一危险,QED仍被列入物理学史上最成功的理论之一,用它预测电子和磁场的作用强度与实验可靠值仅差2/1,000,000,000,000。

尽管QED取得了超凡的成功,它仍然充满谜团。对于虚空空间(真空),理论似乎提供了荒谬的看法,它表明真空不空,它到处充斥着小的电磁涨落。这些小的涨落是解释自发辐射的关键,并且,它们使原子能量和诸如电子等粒子的性质产生可测量的变化。虽然QED是古怪的,但其有效性是为许多已有的最精确的实验所证实的。

对于我们周围的低能世界,量子力学已足够精确,但对于高能世界,相对论效应作用显著,需要更全面的处理办法,量子场论的创立调和了量子力学和狭义相对论的矛盾。

量子场论的杰出作用体现在它解释了与物质本质相关的一些最深刻的问题。它解释了为什么存在玻色子和费米子这两类基本粒子,它们的性质与内稟自旋有何关系;它能描述粒子(包括光子,电子,正电子即反电子)是怎样产生和湮灭的;它解释了量子力学中神秘的全同性,全同粒子是绝对相同的是因为它们来自于相同的基本场;它不仅解释了电子,还解释了μ子,τ子及其反粒子等轻子。

QED是一个关于轻子的理论,它不能描述被称为强子的复杂粒子,它们包括质子、中子和大量的介子。对于强子,提出了一个比QED更一般的理论,称为量子色动力学(QCD)。QED和QCD之间存在很多类似:电子是原子的组成要素,夸克是强子的组成要素;在QED中,光子是传递带电粒子之间作用的媒介,在QCD中,胶子是传递夸克之间作用的媒介。尽管QED和QCD之间存在很多对应点,它们仍有重大的区别。与轻子和光子不同,夸克和胶子永远被幽禁在强子内部,它们不能被解放出来孤立存在。

QED和QCD构成了大统一的标准模型的基石。标准模型成功地解释了现今所有的粒子实验,然而许多物理学家认为它是不完备的,因为粒子的质量,电荷以及其它属性的数据还要来自实验;一个理想的理论应该能给出这一切。

今天,寻求对物质终极本性的理解成为重大科研的焦点,使人不自觉地想起创造量子力学那段狂热的奇迹般的日子,其成果的影响将更加深远。现在必须努力寻求引力的量子描述,半个世纪的努力表明,QED的杰作——电磁场的量子化程序对于引力场失效。问题是严重的,因为如果广义相对论和量子力学都成立的话,它们对于同一事件必须提供本质上相容的描述。在我们周围世界中不会有任何矛盾,因为引力相对于电力来说是如此之弱以至于其量子效应可以忽略,经典描述足够完美;但对于黑洞这样引力非常强的体系,我们没有可靠的办法预测其量子行为。

一个世纪以前,我们所理解的物理世界是经验性的;20世纪,量子力学给我们提供了一个物质和场的理论,它改变了我们的世界;展望21世纪,量子力学将继续为所有的科学提供基本的观念和重要的工具。我们作这样自信的预测是因为量子力学为我们周围的世界提供了精确的完整的理论;然而,今日物理学与1900年的物理学有很大的共同点:它仍旧保留了基本的经验性,我们不能彻底预测组成物质的基本要素的属性,仍然需要测量它们。

或许,超弦理论是唯一被认为可以解释这一谜团的理论,它是量子场论的推广,通过有长度的物体取代诸如电子的点状物体来消除所有的无穷大量。无论结果何如,从科学的黎明时期就开始的对自然的终极理解之梦将继续成为新知识的推动力。从现在开始的一个世纪,不断地追寻这个梦,其结果将使我们所有的想象成为现实。



\section{历史}

\subsection{对光本性的认识}
牛顿微粒说$\rightarrow$ 惠更斯原理(光是纵波)$\rightarrow$ 双缝干涉实验(波)$\rightarrow$ 菲涅耳提出光是横波,解释了光的偏振$\rightarrow$ 麦克斯韦电磁理论,光是电磁波$\rightarrow $赫兹验证之,并发现光电效应(粒子)$\rightarrow$ 康普顿X射线散射(粒子) 

\subsection{波动力学}
爱因斯坦光量子论$\rightarrow$ 德布罗意物质波 $\rightarrow$ 电子衍射实验$\rightarrow$ 薛定谔方程(波动力学建立)$\rightarrow$ 波恩统计解释

\subsection{矩阵力学}
汤姆逊发现电子$\rightarrow$ 卢瑟福$\alpha$粒子散射实验,原子核式结构$\rightarrow $玻尔受巴尔么公式启发,提出氢原子半经典理论$\rightarrow $矩阵力学(海森堡).

\subsection{路径积分}
惠更斯原理$\rightarrow $路径积分(费曼)



\section{量子力学基本概念}

\subsection{波粒二象性 物质波的方程}

 电子、中子和分子等的波动性已被实验证实,这表明粒子具有波粒二象性,它们满足德布罗意关系:
$$E=h\nu=\hbar\omega \eqno 1.1a$$
$$\vec p=\hbar\vec k \eqno 1.1b$$

把它们带入平面波的方程$$\Psi=Ae^{i(\vec k \cdot \vec r-\omega t)}$$
便得到自由粒子的波动方程:
$$\Psi=Aexp\frac{i}{h}(\vec p\cdot\vec r-Et) \eqno1.2 $$






\subsection{不确定性原理}

\ \ 电子的双缝干涉实验中,如使电子一个一个互不影响地通过两条打开的狭缝,得到的干涉图样不是两个狭缝单独衍射时的简单叠加,而是具有波的干涉特性,此与经典观念(每个电子都有自己的轨道,只通过狭缝之一)不符.因此,量子力学中不存在粒子轨道之类的概念.\par
假定对一个电子的坐标测量多次,每次时间间隔$\Delta$t,这些测量结果一般并不位于一条光滑曲线上,相反,测得越准确,反而越不连续,此与电子不存在轨道概念一致。只有在粗略地测量下,如威耳逊云室中,才得到一条光滑轨道.\par
现将$\Delta t$缩短,相邻的测量会给出坐标的相邻值,一系列测量后,所得结果虽都会落到一个很小的空间范围内,但它们毫无规则,因此粒子不能同时有确定的坐标和速度,即不存在确定的轨道,因为坐标和速度确定一条轨道.\footnote{这部分详细内容,请看朗道的书}





\subsection{薛定谔方程}
\ \ 曾谨言的书是以单个自由粒子平面单色波方程为例“推导”的。但是,我们要清楚,推导只是方便理解,薛定谔方程的正确性只能由它导出的结论和实验是否符合来检验,它是五个基本原理之一.

对$$\Psi=Aexp\frac{i}{h}(\vec p\cdot\vec r-Et) \eqno $$两边对t求偏导,得$$ih\frac{\partial}{\partial t}\Psi=E\Psi$$ $$-ih\nabla \Psi=\vec p\Psi$$ $$-\hbar\nabla^{2}\Psi=\vec p^{\ 2}\Psi$$
又因$E=p^{2}/2m$,得$$ih\frac{\partial}{\partial t}\Psi=-\frac{\hbar^{2}\nabla^{2}}{2m}\Psi \eqno 2.1$$
考虑可能存在外场$V_{(\vec r)}$,则$E=p^{2}/2m+V_{(\vec r)}$,代入上式,得薛定谔方程$$ i \hbar \frac{\partial}{\partial t}\Psi(\vec r,t)
= [-\frac{\hbar^2}{2m}\nabla^2+V(\vec r)]\Psi(\vec r,t) \eqno 2.2$$
现在考虑外场V不随时间变化,则$\Psi$可分离变量,$\Psi(\vec r,t)=\psi(\vec r)f(t)$代入2.2式得:$$\frac{ih}{f}\frac{df}{dt}=E \eqno 2.3$$ $$[-\frac{\hbar^2}{2m}\nabla^2+V(\vec r)]\psi(\vec r)=E\psi \eqno 2.4$$ \par 
2.4式常写为$H\psi=E\psi$,H为哈密顿算符,此即定态薛定谔方程.\par







\subsection{叠加原理\ 波函数统计解释}
统一物质的粒子性与波动性这一对矛盾的方法是Born(1926)提出的概率波,这也是唯一的方法.

\textbf{波函数的统计解释},一般量子力学书都有介绍。国内的教材大部分是从双缝干涉引入。但双缝干涉时,粒子穿过两条缝之前需穿过一条狭缝,这样做是为了保证它们的相干性\footnote{国内有学者专门讨论过.费曼物理学讲义中几乎都有前面的这条缝.}.经典的波振幅和能量有关,不能“归一化”;而概率波的振幅和概率有联系,可归一化.

\textbf{态叠加原理}:如果$\Psi_{1}$和$\Psi_{2}$是系统的两个状态,则它们的任意复系数的线性组合$\Psi= c_1\Psi_{1}+c_2\Psi_{2}$也是系统的一个可能状态。但这里的叠加和经典的叠加不同\footnote{狄拉克的《原理》对态叠加原理有精彩论述,推荐大家看一看.}.






\subsection{波函数按动量分布展开}
按照波函数的统计诠释,在空间一点找到粒子的概率$\propto |\Psi|^2(r)$.下面讨论粒子动量的概率分布.

在一般情况下,$\Psi$可看作许多平面单色波叠加而成,即含有各种波长(动量)的分波,因而相应的粒子动量有一个分布$\phi(\vec p)$,与$|\Psi(\vec r)|^2$表示粒子在坐标空间某点的概率密度相似,$|\phi(\vec p)|^2$表示粒子的动量分布的概率密度.$\phi (\vec p)$是$\psi (\vec r)$按平面波的展开$(\Psi (\vec r)$的多重Fourier积分),即$$\Psi (\vec r)=\frac{1}{(2\pi\hbar)^{3/2}}\int \phi(\vec p)exp(i\vec p\cdot\vec r/\hbar)d\vec p \eqno 2.5$$
其逆变换$$\phi(\vec p)=\frac{1}{(2\pi\hbar)^{3/2}}\int \Psi(\vec r)exp(-i\vec p\cdot\vec r/\hbar)d\vec r \eqno 2.6$$
由于薛定谔方程只含波函数对时间的一次微商,只要在初始时刻(t=0)体系的状态$\Psi(\vec r,0)$给定,则以后任何时刻t的状态$\Psi(\vec r,t)$原则上就完全确定了.换言之,薛定谔方程给出了波函数(态)随时间演化的关系.
对自由粒子,$$\Psi (\vec r)=\frac{1}{(2\pi\hbar)^{3/2}}\int \phi(\vec p)exp(i\vec p\cdot\vec r/\hbar-Et)d\vec p \eqno 2.7$$
$\Psi(\vec r,t)$的初态波函数为$$\Psi (\vec r,0)=\frac{1}{(2\pi\hbar)^{3/2}}\int \phi(\vec p)exp(i\vec p\cdot\vec r/\hbar)d\vec p \eqno 2.8$$
$\phi (\vec p)$正是$\Psi(\vec r,0)$的Fourier展开的波幅,它并不依赖于t.上式逆变换为$$\phi(\vec p)=\frac{1}{(2\pi\hbar)^{3/2}}\int \Psi(\vec r)exp(-i\vec p\cdot\vec r/\hbar)d\vec r \eqno 2.9$$把2.9式代入2.7式,得$$\Psi (\vec r,t)=\frac{1}{(2\pi\hbar)^{3/2}}\int d\vec r_{0}\int d\vec p \phi(\vec p)e^{\frac{i}{\hbar}[\vec p\cdot(\vec r-\vec r_{0})/-Et]}\Psi(\vec r_0,0)\eqno 2.10$$这样,体系的初始状态完全决定了以后任何时刻t的状态.更一般讲,取初始时刻为$t_0$,则$${\begin{aligned}\Psi (\vec r,t) &=\frac{1}{(2\pi\hbar)^{3/2}}\int d\vec r_{0}\int d\vec p \phi(\vec p)e^{\frac{i}{\hbar}[\vec p\cdot(\vec r-\vec r_{0})-E(t-t_0)]}\Psi(\vec r_0,t_0)\\
&=\int d\vec r_{0}G(\vec r,t;\vec r_0,t_0)\Psi(\vec r_0,t_0)\end{aligned}} \eqno 2.11$$
式中的$$G(\vec r,t;\vec r_0,t_0)=\frac{1}{(2\pi\hbar)^{3/2}}\int d\vec p \phi(\vec p)e^{\frac{i}{\hbar}[\vec p\cdot(\vec r-\vec r_{0})-E(t-t_0)]}$$称为传播子.借助于传播子G,体系在时刻t的状态可由时刻$t_0时$的状态给出。2.11式可类比光学中的惠更斯原理。


\subsection{连续性方程}

\ \ 在非相对论情况下,实物粒子没有产生和湮没的现象, 所以在随时间演化的过程中,粒子数目保持不变.对于一个粒子来说,在全空间中找到它的概率之总和应不随时间改变,即$$\frac{d}{dt}\int \Psi^{*}\Psi d \tau =0 $$

这可用薛定谔方程得出:

对薛定谔方程取复共轭,注意 V*= V ,得$$ -i \hbar \frac{\partial}{\partial t}\Psi(\vec r,t)= [-\frac{\hbar^2}{2m}\nabla^2+V(\vec r)]\Psi^{*}(\vec r,t) \eqno 2.4$$
 由$\Psi^{*}×(2.2)-\Psi×(2.4)$,得
$$\-i \hbar \frac{\partial}{\partial t}\Psi(\vec r,t)\\ = -\frac{\hbar^{2}}{2m}(\Psi^*\nabla^2\Psi-\Psi\nabla^2\Psi^*) \\ =-\frac{\hbar^{2}}{2m}\nabla\cdot(\Psi^*\nabla\Psi-\Psi\nabla\Psi^* ) $$
令$$\rho=\Psi^*\Psi $$,$$\vec j=-\frac{\hbar}{2m}\nabla\cdot(\Psi^*\nabla\Psi-\Psi\nabla\Psi^{*})$$ 
则在空间 闭 区 域V中 将 2.5式 积 分, 并由高斯定 理可得,$$\frac{d}{dt}\int\rho d\tau=-\oint_S \vec j \cdot d \vec S$$

上式左边代表在闭区域$\tau$中找到粒子的总概率(或粒子数)在单位时间内的增加,而右边表示单位时间内通过$\tau$的封闭表面S而流入$\tau$内的概率(注意负号!).所以$\vec j$具有概率流(粒子流)密度的意义,是一个矢量.上式是概率(粒子数)守恒的积分表示式,微分式为$$\frac{\partial}{\partial t}\rho +\nabla \cdot \vec j=0$$ 其形式与流体力学中的连续性方程相同.
让$\tau $为全空间,可以证明2.5式右侧的积分趋于零,所以$$\frac{d}{dt}\int \Psi^{*} \Psi d\vec r=0 $$
即归一化不随时间而改变.在物理上这表示粒子既未产生, 也未湮没.这里的概率守恒具有定域的性质.当粒子在空间某地 的概率减小了,必然在另外一些地方的概率增加了(使总概率不变),并且伴随着有什么东西在流动,来实现这种变化,连续性就意味着某种流的存在.




\section{力学量用算符表示}

\subsection{算符的本征值与本征方程}
\ \ 量子力学的主要数学工具是希尔伯特空间\footnote{推荐大家在学习算符和矩阵之前学会。}。在这个线性空间里可以进行加法,数乘和内积三种运算.\par
\textbf{基本原理1 描写微观系统状态的数学量是希尔伯特中间中的矢量,相差一个复数因子的两个矢量,描写同一个状态.}

这里,我们用抽象的右矢$\ket{\Psi}$(dirac符号)或者左矢$\bra{\Psi}$来表示微观系统的状态,不涉及任何具体的表象.
如果一个空间中的一组线性无关的矢量,它们的线性叠加可以表示空间中任一个矢量,则这组矢量为此空间的一组\textbf{完全集}.正交归一化的完全集称为此空间的一组\textbf{基矢}。

如果映射A满足$\ket{\psi}=A\ket{\phi}$,则A称为\textbf{算符}.我们说,算符A作用于右矢$\ket{\phi}$得到右矢$\ket{\psi}$.考虑对称性便有必要引进\textbf{逆算符}和\textbf{伴算符}.

如果算符$A^{-1}$满足$A^{-1}\ket{\psi}=\ket{\phi}$,且算符A使关系$\ket{\psi}=A\ket{\phi}$一一对应,则$A^{-1}$称为算符A的\textbf{逆算符},显然,$$A^{-1}A=AA^{-1}=1$$

与右矢相对应,左矢空间中$\bra{\psi}=\bra{A\phi}=A^{\dagger}\bra{\phi}$,这里$A^{\dagger}$即为A的\textbf{伴算符}.如A为复数算符a,则其伴算符为a*.

算符乘积的定义是:$$BA\ket{\psi}=B(A\ket{\psi})$$如果两个算符满足AB=BA,我们就说这两个算符是可\textbf{对易}的,各个算符之间并非都是可对易的,因此,我们用$$[A,B]=AB-BA$$来表示两个算符的对易关系.

\textbf{基本原理3 微观系统中每个粒子的直角坐标下的位置算符X与相应的正则动量算符P有如下对易关系:$$[X_i,X_j]=0,[P_iP_j]=0,[X_i,P_j]=i\hbar\delta_{ij}$$不同粒子间的所有算符均互相对易.}

对于算符A,若有非零矢量$\ket{\psi}$满足下式:$$A\ket{\psi}=a\ket{\psi}$$式中$a$为数,则$\ket{\psi}$称为算符A的\textbf{本征矢量},$a$称为相应的\textbf{本征值},上式为\textbf{本征值方程}.

\subsection{厄米算符}
对任意的态矢量$\ket{\psi}$ 和$\ket{\phi}$,如果算符A满足$\braket{\psi|A\phi}=\braket{A\psi|\phi}$,即$A^\dagger=A$,则A就是\textbf{厄米算符}.

厄米算符的基本性质:
1 厄米算符的本征值必为实.
2 厄米算符的属于不同本征值的本征函数彼此正交.

\textbf{基本原理2  描写微观系统物理量的是希尔伯特空间中的厄米算符;物理量所能取的值,是相应算符的本征值;物理量A在状态$\ket{\psi}$中取各值$a_i$的概率,与态矢量按A的归一化本征矢量\{$\ket{a_i}$\}的展开式中$\ket{a_i}$的系数的复平方成正比,即与下式中$c_i$的复平方成正比:$$\ket{\psi}=\sum_i\ket{a_i}c_i,c_i=\braket{a_i|\psi}$$}

会求具体的力学量的算符的本征值和本征函数\footnote{今年CAS就考了角动量$\hat{L_z}$},如动量算符$\hat{\vec{p}}$,角动量算符$\hat{L^2}$,角动量算符$\hat{L_z}$。


\subsection{力学量随时间的演化}
\ \ 量子力学中力学量随时间演化的问题,与经典力学有所不同.经典力学中,处
于一定状态下的体系的每一个力学量A ,作为时间的函数,在每一时刻都具有一个确定值.量子力学中,处于量子态$\psi$下的体系,在每一时刻,不是所有力学量都具有确定值,一般说来,只具有确定的概率分布和平均值.

先讨论力学量的平均值如何随时间改变.力学量A 的平均值为
$$\bar{A}=\frac{\sum_{i}|c_i|^2a_i}{\sum_{i}|c_i|^2}=\frac{\sum_{i}\braket{\psi|a_i}\braket{a_i|\psi}a_i}{1}=\sum_{i}\braket{\psi|A|a_i}\braket{a_i|\psi}=\braket{\psi|A|\psi} \eqno 3.1$$

利用薛定谔方程和H的厄米性,可得$$\frac{d\bar{A}}{dt}=\braket{\frac{H\psi}{i\hbar}|A\psi}+\braket{\psi|A\frac{H\psi}{i\hbar}}+\braket{\psi|\frac{\partial A}{\partial t}\psi}=\frac{1}{i\hbar}\overline{[A,H]}+\frac{\overline{\partial A}}{\partial t}$$

如A不显含t,即$\frac{\partial A}{\partial t}=0$,则$$\frac{d\bar{A}}{dt}=\frac{1}{i\hbar}\overline{[A,H]}$$
对于Hamilton量H不含时的量子体系,如果力学量A 与H对易,则无论体系处于什么状态(定态或非定态), A的平均值及其测值的概率分布均不随时间改变.所以把A称为量子体系的一个守恒量.


\subsection{不确定度关系}
当体系处于力学量A的本征态时, 若对它测量A,则可得到一个确切值,即相应的本征值,而不会出现涨落. 若在A的这个本征态下去测量另一个力学量B,是否也能得到一个确定的值 ? 不一定. 如,粒子的位置与动量不能同时完全确定。

任意两个力学量A和B都满足关系:$$\Delta A\Delta B \geq \frac{1}{2}| \overline{[A,B]}| $$

可以看出,若两个力学量A与B不对易,则一般说来 ΔA与ΔB不能同时为零,即A与B不能同时测定,或者说它们不能有共同本征态.反之,若两个厄密算符对易,则可以找出使ΔA=0与ΔB=0同时满足的态,即可以找出它们的共同本征态.如,$l^2$和$l_z$.










\section{表象理论}
\subsection{力学量算符的矩阵表示}
现在,我们用一组数来具体地表示矢量和算符.

为此,首先在矢量空间中选定一组基矢${\ket{i}}$,考虑离散情况,取有物理意义的厄米算符构成对易完备组K,用它们的共同本征矢量作为作为基矢$$K\ket{i}=k_i\ket{i}$$单一字母i表示算符完备组中各算符本征值序号的集合.取定这样的一组基矢称为取一个表象(K表象).

空间中任意矢量都可以按这组基矢展开:
$$\ket{\psi}=\sum_i \ket{i}\braket{i|\psi}=\sum_i\ket{i}\psi_i,\psi_i=\braket{i|\psi}$$
$$\ket{\phi}=\sum_j \ket{j}\braket{j|\psi}=\sum_j\ket{j}\psi_j,\phi_j=\braket{j|\phi}$$
式中$\psi_i$称为矢量$\psi$在基矢$\ket{i}$上的分量,这是一个复数.为了具体地表示一个矢量$\ket{\psi}$,指出它在选定的这组基矢上的全部分量${\psi_i}$即可.
设A是一个确定的算符,有$\ket{\phi}=A\ket{\psi}$,用基左矢对其两边作内积,利用完全性关系得
$$\braket{j|\phi}=\sum_i\braket{j|A|i}\braket{i|\psi}$$
对于确定的算符A,则上式为$$\phi_i=\sum_i A_{ji}\psi_i$$
于是,知A和$\ket{\psi}$就可以算出$\ket{\phi}$.

使用矩阵,对右矢我们有:

$$\ket{\psi}\rightarrow\begin{pmatrix}\psi_1\\\psi_2\\ \vdots \\\psi_n\end{pmatrix}
,\ket{\phi}\rightarrow\begin{pmatrix}\phi_1\\\phi_2 \\ \vdots \\\phi_n\end{pmatrix}$$
对左矢有:$$\bra{\psi}\rightarrow\begin{pmatrix}\psi_1^*&\psi_2^*\cdots \psi_n^*\end{pmatrix},\bra{\phi}\rightarrow\begin{pmatrix}\phi_1^*&\phi_2^*\cdots \phi_n^*\end{pmatrix}$$
对算符有:$$A\rightarrow\begin{pmatrix}A_{11}&A_{12}&\cdots &A_{1n}\\A_{21}&A_{22}&\cdots &A_{2n}\\\cdots&\cdots&\cdots &\cdots\\A_{n1}&A_{n2}&\cdots &A_{nn}\end{pmatrix}$$
对于$A\ket{\psi}=\lambda\ket{\psi}$,则
$$\begin{pmatrix}A_{11}&A_{12}&\cdots &A_{1n}\\A_{21}&A_{22}&\cdots &A_{2n}\\\cdots&\cdots&\cdots &\cdots\\A_{n1}&A_{n2}&\cdots &A_{nn}\end{pmatrix}\begin{pmatrix}\psi_1\\\psi_2\\ \vdots \\\psi_n\end{pmatrix}=\lambda\begin{pmatrix}\psi_1\\\psi_2\\ \vdots \\\psi_n\end{pmatrix}$$
将等号右边移到左边,得:$$\begin{pmatrix}A_{11}-\lambda&A_{12}&\cdots &A_{1n}\\A_{21}&A_{22}-\lambda&\cdots &A_{2n}\\\cdots&\cdots&\cdots &\cdots\\A_{n1}&A_{n2}&\cdots &A_{nn}-\lambda\end{pmatrix}\begin{pmatrix}\psi_1\\\psi_2\\ \vdots \\\psi_n\end{pmatrix}=0$$
上式有解需:$$\begin{vmatrix}A_{11}-\lambda&A_{12}&\cdots &A_{1n}\\A_{21}&A_{22}-\lambda&\cdots &A_{2n}\\\cdots&\cdots&\cdots &\cdots\\A_{n1}&A_{n2}&\cdots &A_{nn}-\lambda\end{vmatrix}=0$$



\subsection{幺正变换}
\par 一个空间中有许多不同的基,因此,矢量和算符也有不同表象.

满足:$A^{\dagger}A=AA^{\dagger}=I$的算符,定义为幺正算符.

一,力学量的变换:

在矢量空间中,若\{$\ket{v_i}$\}是一组基矢,则\{$U\ket{v_i}$\}也是一组基矢.是因为:1,$\braket{Uv_i|Uv_j}=\braket{v_i|v_j}=\delta_{ij}$,即\{$U\ket{v_i}$\}满足正交归一化条件.2,$\sum_i\braket{\psi|Uv_i}\braket{Uv_i|\phi}=\sum_i\braket{U^\dagger\psi|v_i}\braket{v_i|U^\dagger\phi}=\braket{U^\dagger\psi|U^\dagger\phi}=\braket{\psi|\phi}$,即完全性.

于是,对状态$\ket{\psi}$和$\ket{\phi}$有:$$\ket{\psi'}=U\ket{\psi},\ket{\phi'}=U\ket{\phi}$$

二,算符的变换:
$\ket{\psi}=A\ket{\phi}$,相应的幺正变换为$\ket{\psi'}=A'\ket{\phi'}$,下面求A'与A的关系.
 $$\ket{\phi'}=U\ket{\phi}=UA\ket{\psi}=UAU^{-1}\ket{\psi'}=A'\ket{\psi'}$$
 即$$A'=UAU^{-1}$$

算符对一个态的作用,即测量这个算符所对应的力学量,还是在原来的表象。而幺正变换是通过改变基矢从一个表象变换到另一个
表象,

 
 
 \section{一维势场}



\subsection{一般性质}
如果一个粒子的势能只依赖于一个坐标x,则波函数可表示X(x)Y(y)Z(z).X(x)满足一维薛定谔方程:$$\frac{d^2\psi}{dx^2}+\frac{2m}{\hbar^2}[E-V(x)]\psi=0$$ 

如果V(x)是规则势场,如存在束缚态,则必定是不简并的.

如果V(x)是偶函数,即V(x)=V(-x),如$\psi(x)$是方程的一个解,则$\psi(-x)$也是一个解.并且只差一个常数因子:$\psi(-x)=c\psi(x)$.再把x反号一次,得$\psi(x)=c^2\psi(x),c=\pm 1$.因此,对于具有对于x=0对称的势能,定态波函数只能是偶函数或奇函数.

振荡定理:对离散谱的波函数,属于第(n+1)个能级$E_n$的本征函数$\psi_n(x)$共有n次等于零(对x为有限值而言).

如势阱突然扩大,波函数未来得及变化,但算符$\hat{H}$已瞬间变化。

\subsection{方势阱}

势阱:$V(x)=0,0<x<a;V_0,else$.在$0<x<a$区域,$$\frac{d^2\psi}{dx^2}+\frac{2m}{\hbar^2}E\psi=0$$在else区域,$$\frac{d^2\psi}{dx^2}+\frac{2m}{\hbar^2}[E-V_0]\psi=0$$ 
在x=0和x=a点,波函数及其导数均连续(势场的变化$\Delta V$为有限时,波函数及其对x的导数均连续),上式的解对$x\rightarrow \pm\infty$时必须保持有限.

对$E<V_0$,阱外满足在无穷远处为零解为(正负号按条件选取)$$\psi=A\times e^{\pm kx},k=\frac{1}{\hbar}\sqrt{2m(V_0-E)}$$
势阱边上的连续性条件,可撇开归一化问题,用其对数导数$\psi'/\psi$的连续性条件来代替,求得$$|\psi'|/\psi=\pm k$$

当$V_0=\infty$时,在x=0和x=a点的波函数为零. 阱内,$$\psi=c\sin kx,k=\frac{\sqrt{2mE}}{\hbar}$$
在x=a处需$ka=n\pi$,得$$E=\frac{\psi^2\hbar^2n^2}{2ma^2},n=1,2,3,\cdots$$
$$\psi_n=\sqrt{\frac{2}{a}}\sin(\pi nx/a)$$



\subsection{势垒}
设具有一定能量 E 的粒子沿 x 轴正方向射向方势垒:$$V(x)=\left\{\begin{array}{lc}{V_{0} }&{ 0<x<a}\\{0 }&{ else}
\end{array}\right.$$
按照波函数的统计诠释,无论粒子能量$E>V_0$ ,或$E<V_0$ ,都有一定概率穿透势垒,也有一定概率被反射回去.
考虑$E<V_0$情况.在势垒外($x<0, x>a$):$$\frac{d^2\psi}{dx^2}=-k^2\psi,k=\sqrt{2mE}/\hbar$$
它的两个线性无关解可取为$\psi=A\times e^{\pm kx}$.以下假设, 粒子是从左入射.由于势垒的存在, 在$x<0$区域, 既有入射波,也有反射波.而在 $x > a$区域中则只有透射波.所以,
$$\psi(x)=\left\{\begin{array}{lc}{e^{ike}+Re^{-ike}}{,x<0}\\{Se^{ike}}{,x>a}\end{array}\right.
$$上式中入射波的波幅取为1,只是为了方便 (这对反射和透射系数无影响),则
$$\begin{array}{c}{F
=|R|^2}\\{ D
=|S|^2}
  \end{array}
$$F为反射系数,D为透射系数.

在势垒内部考虑在$x=0$和$x=a$点,$\psi$与$\psi'$的连续性条件,可解得F和D.如入射能量合适,会发生共振透射.

粒子能穿透比它动能更高的势垒的现象,称为隧穿效应 (tunnel effect), 它是粒子具有波动性的表现.
隧穿效应实例:$\alpha$衰变,热核聚变,STM.

共振透射可和力学中的驻波类比学习。


\subsection{$\delta$ 势}
如粒子从左入射,遇到$\delta$势垒$V(x)=\gamma\delta(x)$,薛定谔方程为$$-\frac{\hbar^2}{2m}\frac{d^2}{d^2x}\psi=[E-\gamma\delta(x)]\psi$$对上式积分$\int^{+\varepsilon}_{-\varepsilon}dx,\varepsilon\rightarrow 0$,由$\delta$函数积分的挑选性,可得$$\psi'(0^+)-\psi'(0^-)=\frac{2m\gamma}{\hbar^2}\psi(0)$$因此,$\psi'$在x=0点是不连续的,上式对处理$\delta$势场问题中起关键作用.

\subsection{谐振子}
振动往往可以近似分解为一维谐振动. 取自然平衡位置为原点,规定其为势能零点,则一维谐振子的势能可以表示为$$V(x)=\frac{1}{2}Kx^2$$
令$\omega=\sqrt{K/m}$,则一维谐振子的哈密顿量为$$H=\frac{p^2_x}{2m}+\frac{1}{2}m\omega_0^2x^2$$
解得$E_n=(n+\frac{1}{2})\hbar\omega_0,n=0,1,2,...$


下面介绍薛定谔提出哈密顿量的因式分解法\footnote{Griffiths的书论述的比较好}.

令$$\hat{a}=\sqrt{\frac{\mu\omega}{2\hbar}}(\hat{x}+\frac{i}{\mu\omega}\hat{p})$$,$$\hat{a}^+=\sqrt{\frac{\mu\omega}{2\hbar}}(\hat{x}-\frac{i}{\mu\omega}\hat{p})$$则可求得$$\hat{p},\hat{x}$$

$$[\hat{a},\hat{a}^+]=1$$令$$\hat{N}=\hat{a}^+\hat{a}$$则$$H=(\hat{N}+\frac{1}{2})\hbar\omega$$
$$\hat{a}\ket{n}=[\hat{a},\hat{a}^+]\hat{a}\ket{n-1}=(\hat{a}\hat{N}-\hat{N}\hat{a})\ket{n}=\hat{a}n\ket{n}-\hat{N}\hat{a}\ket{n}$$

得$$\hat{N}\hat{a}\ket{n}=(n-1)\hat{a}\ket{n}$$于是,$$\hat{a}\ket{n}=\lambda\ket{n-1},\bra{n}\hat{a}^+=\lambda^*\bra{n-1}$$得$$\lambda^2=\bra{n}|\hat{a}^+\hat{a}|\ket{n}=n$$所以,$$\lambda=\sqrt{n}$$ $$\hat{a}\ket{n}=\sqrt{n}\ket{n-1}$$ N的本征值$$n=\bra{n}|\hat{N}|\ket{n}=\bra{\psi}|\ket{\psi}>=0$$
由以上两个方程,得n=0,1,2,...则H的本征值为$$E_n=(n+\frac{1}{2})\hbar\omega$$
$$\hat{a}^+\ket{n}=\sqrt{n+1}\ket{n+1}$$
$$\ket{n}=\frac{1}{\sqrt{n!}}(\hat{a}^+)^n\ket{0}$$ 
在x表象中,$$\vec{\hat{p}}=-i\hbar\frac{d}{dx}$$
代入上式可求$\psi_{n}(x)$
而$$\hat{a}\ket{0}=0$$联立,即可解得$\psi_{n}(x)$。




\section{自旋}

\subsection{电子的自旋}
实验分析表明,电子不是一个只具有坐标空间的三个自由度的粒子, 它还具有
一个内禀由度—自旋. 要对它的状态作出完全的描述, 还须考虑其自旋状态,即要考虑自旋在某给定方向(例如z轴方向)的投影的两个可能取值的波幅,即波函数中还应包含自旋投影这个变量.




\subsection{自旋算符}
类比角动量
考虑到自旋具有角动量的特征,假设自旋s的三个分量具有与轨道角动量l的三个分量相同的对易关系.令$S=\frac{\hbar}{2}\sigma$,则$$
\begin{vmatrix}
	x& y & z\\
	\sigma_x&\sigma_y&\sigma_z\\
	\sigma_x&\sigma_y&\sigma_z\\
\end{vmatrix}=2i(\sigma_x,\sigma_y,\sigma_z)$$
上式即$\vec{\sigma}\times\vec{\sigma}=2i\vec{\sigma}$.

选择具体的$\sigma_z$表象,即$\sigma_z$对角化的表象,所以$$\sigma_z=\begin{bmatrix}
	1 & 0\\
	0& -1\\
\end{bmatrix}$$
由以上关系和$\sigma_x$的厄米性,$$\sigma_x=\begin{bmatrix}
	0& e^{i\alpha}\\
	e^{-i\alpha}& 0\\
\end{bmatrix}, 
\sigma_y=-i\begin{bmatrix}
	0& e^{i\alpha}\\
	e^{-i\alpha}& 0\\
\end{bmatrix}$$
态有一个相位的不定性,泡利取$\alpha=0$,得泡利矩阵:$$\sigma_x=\begin{bmatrix}
	0& 1\\
	1& 0\\
\end{bmatrix},
\sigma_y=\begin{bmatrix}
	0& -i\\
	-i& 0\\
\end{bmatrix},
\sigma_x=\begin{bmatrix}
	1& 0\\
	0& 1\\
\end{bmatrix}$$


\subsection{自旋单态与三重态}
考虑两个电子的自旋算符分别为$S_1,S_2$,令$S=S_1+S_2$,易知$[S_1,S_2]=0,[S^2,S_\alpha]=0,\alpha=x,y,z.$两个电子体系的自旋自由度为2,可选$(S_{1z},S_{2z})or (S^2,S_z)$为自旋力学量的完全集.$(S_{1z},S_{2z})$的共同本征态有4个,即$$\alpha(1)\alpha(2),\beta(1)\beta(2),\alpha(1)\beta(2),\beta(1)\alpha(2)$$
$\alpha(1)\alpha(2)$和$\beta(1)\beta(2)$是$S_z$的不简并本征态,且$[S^2,S_z]=0$可以判断$\alpha(1)\alpha(2)$和$\beta(1)\alpha(2)$也是$S^2$的本征态,本征值为 $2\hbar^2$.$S^2$的另外两个本征态可以从$S_z$的两个简并的本征态$\alpha(1)\beta(2)$和$\beta(1)\alpha(2)$线性叠加构成.$\chi=c_1\alpha(1)\beta(2)+c_2\beta(1)\alpha(2)$,可求出$S^2$的归一化本征态为$$\frac{1}{\sqrt{2}}\alpha(1)\beta(2)\pm\beta(1)\alpha(2)$$
以上两态即纠缠态.



\section{中心力场}
\subsection{两体问题\ 径向方程}

\ \ 两个质量分别为$m_1,m_2$的粒子.能量本征方程为$$[-\frac{\hbar^2}{2m_1}\nabla_1^2-\frac{\hbar^2}{2m_2}\nabla_1^2+V|r_1-r_2|]\Psi(r_1,r_2)=E\Psi(r_1,r_2) \eqno 6.1$$
引进质心坐标$\vec R$和相对坐标$\vec r$
$$\vec r=\vec r_1-\vec r_2,R=\frac{m_1\vec r_1+m_2\vec r_2}{m_1+m_2}$$则
$$\frac{1}{m_1}\nabla_1^2+\frac{1}{m_2}\nabla_2^2=\frac{1}{M}\nabla_R^2+\frac{1}{\mu}\nabla^2$$其中$M=m_1+m_2$(总质量),$\mu=m_1m_2/(m_1+m_2)$(约化质量).故6.1式化为$$[-\frac{\hbar^2}{2M}\nabla_R^2-vf\frac{\hbar^2}{2\mu}\nabla^2+V(r)]\Psi=E\Psi$$
令$\Psi=\Phi(R)\psi(r)$,得$$-\frac{\hbar^2}{2m}\nabla^2_R\Psi(R)=E_c\psi$$
$$(-\frac{\hbar^2}{2\mu}\nabla^2+V(r))\psi(r)=(E-E_c)\psi(r)$$

径向方程,钱伯初的书比较直观,我在这就不展开了.

\subsection{氢原子 类氢原子}
氢原子能谱$$E_n=-\frac{\mu e^4}{2\hbar^2}\frac{1}{n^2}=-\frac{e^2}{2a}\frac{1}{n^2},a=\frac{\hbar^2}{\mu e^2},n=1,2,3...$$
类氢离子能谱$$E_n=-\frac{\mu e^4}{2\hbar^2}\frac{Z^2}{n^2}=-\frac{e^2}{2a}\frac{Z^2}{n^2},a=\frac{\hbar^2}{\mu e^2},n=1,2,3...$$这里的Z即原子核所带正电荷数.
类氢离子势场$V(r)=-\frac{Ze^2}{r}$中,定态波函数$$\psi_{nlm}(r)=R_{nl}(r)Y_{lm}(\theta,\varphi)$$
$$R_{nl}(r)=N_{nl}e^{-\frac{Zr}{na}}(\frac{2Zr}{na})^lF(l+1-n,2l+2,\frac{2Zr}{na})$$
$$N_{nl}=\frac{2}{a^{3/2}n^2(2l+1)!}\sqrt{\frac{(n+l)!}{(n-l-1)!}},n=1,2,3...$$
其中a是波尔半径,F是合流超几何函数,基态时(n=1),$R_{10}=\frac{2}{a^{3/2}}e^{-r/a}$

\section{近似方法}
\subsection{非简并定态微扰}
体系的能量本征值往往不能严格求解,常用合适的近似解法,如微扰,变分法,准经典近似等,应用最广的是微扰.

假设系统的哈密顿量可分解为$H=H_0+H'$,$H_0$一般较易求出,H'是一个小量,称为微扰.
因此,可以在$H_0$的本征解的基础上,把H'的影响逐级考虑进去,按微扰进行逐级展开.
$$\ket{\psi}=\ket{\psi^{(0)}}+\ket{\psi^{(1)}}+\ket{\psi^{(2)}}+...$$
$$E=E^{(0)}+E^{(1)}+E^{(2)}+...$$
$$ \therefore (H_0+H')(\ket{\psi^{(0)}}+\ket{\psi^{(1)}}+\ket{\psi^{(2)}}+...)$$$$=(E^{(0)}+E^{(1)}+E^{(2)}+...)(\ket{\psi^{(0)}}+\ket{\psi^{(1)}}+\ket{\psi^{(2)}}+...)$$

零级,$$(H_0-E^{0})\ket{\psi^{(0)}}=0$$

一级,$$(H_0-E^{0})\ket{\psi^{(1)}}=(E^{(1)}-H')\ket{\psi^{(0)}} \eqno 7.2$$

二级,$$(H_0-E^{0})\ket{\psi^{(2)}}=(E^{(1)}-H')\ket{\psi^{(1)}}+E^{(2)}\ket{\psi^{(0)}}$$

约定波函数各级高级近似解与零级近似解都正交,即$\braket{\psi^{(0)}|\psi^{(s)}}=0,s=1,2,3...$
则$$E^{(1)}=\braket{\psi^{(0)}|H'|\psi^{(0)}}$$
$$E^{(2)}=\braket{\psi^{(0)}|H'|\psi^{(1)}}$$
假设在补考虑微扰时,体系处于非简并能级$E_k^{(0)}$,即$E^{(0)}=E_k^{(0)}$,相应的零级能量本征函数完全确定,$\ket{\psi'^{(0)}}=\ket{\psi_k^{(0)}}$

\underline{一级近似}

$$Usually \ \ket{\psi{(1)}}=\sum_n a_n^(1)\ket{\psi_n^{(0)}}$$
$$ \therefore (H_0-E{(0)})\sum_n a_n^(1)\ket{\psi_n^{(0)}}=(E^{(1)}-H')\ket{\psi_k^{(0)}}$$
两边左乘$\bra{\psi_m^{(0)}},$利用$H_0$本征态的正交归一性有
$$\sum_n a_n^{(1)}\braket{\psi_m^{(0)}|H_0|\psi_n^{(0)}}-\sum_n a_n^{(1)}\braket{\psi_m^{(0)}|E_k{(0)}|\psi_n^{(0)}}
$$$$=E^{(1)}\delta_{mk}-H'_{mk}$$
$$a_m^{(1)}(E_m^{(0)}-E_k^{(0)})=E^{(1)}\delta_{mk}-H'_{mk}$$

$m=k$时,$E^{(1)}=E_k^{(1)}=H'_{kk}=\braket{\psi_k^{(0)}}|H'|\psi_k^{(0)}$

$m\neq k$ 时,$a_m^{(1)}=\frac{H'_{mk}}{E_k^{(0)}-E_m^{(0)}},(m\neq k)$

能量$E=E^{(0)}+H'_{kk}$

波函数$ket{\psi_k}=\ket{\psi_k^{(0)}}+\ket{\psi_k^{(1)}}=\ket{\psi_k^{(0)}}+\sum_n'\frac{H'_{nk}}{E_k^{(0)}-E_n^{(0)}}\ket{\psi_n^{(0)}}$

\underline{二级近似}
$$E^{(2)}=E_k^{(2)}\braket{\psi_k^{(0)}|H'|\psi_k^{(1)}}=\sum_n\frac{|H'_{nk}|^2}{E_k^{(0)}-E_n^{(0)}}$$
$$\therefore \ E_k=E_K^{(0)}+H'_{kk}+\sum_n'\frac{|H'_{nk}|^2}{E_k^{(0)}-E_n^{(0)}}$$

\subsection{简并微扰}
假设$E_n^{(0)}$是k度简并.零级近似波函数$\psi_n^{(0)}$写成k个$\phi_i$的线性组合$$\ket{\psi_n^{0}}=\sum_{i=1}^kc_i^{(0)}\ket{\phi_i}$$代入7.2式
$$(H_0-E^{0})\ket{\psi^{(1)}}=(E^{(1)}-H')\sum_{i=1}^kc_i^{(0)}\ket{\phi_i}$$
左乘$\bra{\phi_l}$得,$$0=E^{(1)}\sum_i^kc_i\delta_{il}-\braket{\phi_l|H'|\phi_i}\sum_i^kc_i \eqno 7.3$$
上式是系数$c_i^{(0)}$为未知量的一次齐次方程组.
久期方程:$$det|H'_{li}-E_n^{(1)}\delta_{li}|=0$$

解此方程可得能量的一级修正$E_n^{(1)}$的k个根.因此$E_n=E_n^{(0)}+E_n^{(1)}$,若$E_n^{(1)}$的k个根都不相等,则一级微扰可以将k度简并完全消除;若$E_n^{(1)}$有几个重根,说明能级简并未完全解除.须考虑能量的二级修正,才有可能使能级完全裂开.



\subsection{变分 }

\ \ 推荐周世勋的教材.\footnote{我一个同学说,看曾老的变分法,愣是没看懂是什么意思.呵呵,我也没看懂...}先对能量本征函数作某种限制,即选择某种在数学形式上比较简单, 在物理上也较合理的试探波函数$\Psi_{(\lambda)}$, 然后给出该试探波函数形式下的能量平均值$\overline{H}$,并让$\overline{H}$取极值, 从而定出在所取形式下的最佳的能量本征函数,平均值应该大于等于基态能量。

\section{多体问题 二次量子化}

\subsection{全同性原理 泡利不相容原理}
\ \ \textbf{基本原理5\ 描写全同粒子系统的态矢量,对于任意一对粒子的对调是对称或反对称的服从前者的粒子为波色子,服从后者的为费米子.}

\textbf{泡利不相容原理:不能有两个全同的费米子处于同一个单粒子态}

用$P_{ij}$表示第i个粒子与第j个粒子交换,交换一次,$P_{ij}\psi=\lambda\psi$,再交换一次$P_{ij}^2\psi=\lambda^2\psi$,由于$P^2_{ij}=1$,所以$\lambda^2=1,\lambda=\pm 1$



\newpage 
\end{CJK}

\end{document}